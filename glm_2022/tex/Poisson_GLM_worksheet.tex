\documentclass[a4paper,12pt]{article}
\usepackage[utf8x]{inputenc}
\usepackage[top=2cm, left=2cm, right=2cm, bottom=2cm]{geometry}
\usepackage{times}
\usepackage{enumitem}
\usepackage{hyperref}
\usepackage{color}
\usepackage{amsmath}
\usepackage{longtable}
\setitemize{itemsep=0pt, topsep=0pt}
\newcommand{\mydash}{\hspace{0.2cm}--\hspace{0.2cm}}
%opening
\title{Poisson GLM worksheet}
\author{C\'oil\'in Minto, Olga Lyashevska}
\date{}
\begin{document}

\maketitle
\noindent Jellyfish populations are an important contemporary topic in ocean sciences. To study the effect of temperature on jellyfish abundance, a researcher looked at the number of \emph{Aurelia spp.} per square kilometre in the Northeast and Northwest Atlantic over the course of a season. The temperature of each observation was also recorded. The researcher is interested in the effects of are and temperature on jellyfish abundance.\\
The file \texttt{jellyfish.txt} is \href{https://github.com/mintoc/mfrc-stats/tree/main/glm_2022/data}{here}.
\vspace{1cm}
\begin{itemize}
 \item[{\bf 1.}] Read the data into R.
 \item[{\bf 2.}] Plot the data in a way you think suitable.
 \item[{\bf 3.}] Use the \texttt{glm} function in R to fit Poisson regressions (log-linear models) with:
\begin{itemize}
\item[a.] Just a grand mean
\item[b.] A grand mean plus a grand effect of temperature
\item[c.] By-area mean
\item[d.] By-area mean plus a grand effect of temperature
\item[e.] By-area mean plus by-area effect of temperature
\end{itemize}
 \item[{\bf 5.}] Can you interpret what the parameters mean?
 \item[{\bf 6.}] Overlay your plot with predictions from one of the models
 \item[{\bf 7.}] Bonus: add 95\% confidence intervals to your lines
\end{itemize}

\end{document}
