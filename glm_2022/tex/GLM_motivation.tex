\documentclass[dvipsnames]{beamer}

\usepackage[utf8]{inputenc}
\usetheme{default}

\useinnertheme{circles}

%% set colours
\definecolor{atugreen}{HTML}{005b5e}
\setbeamercolor{title}{fg=atugreen}
\setbeamercolor{frametitle}{fg=atugreen}
\setbeamercolor{section in toc}{fg=atugreen}
\setbeamercolor{itemize item}{fg=atugreen}
\setbeamercolor{itemize subitem}{fg=orange}
\setbeamertemplate{section in toc}{\inserttocsectionnumber.~\inserttocsection}
\setbeamercolor{section number projected}{bg=white,fg=atugreen}

\title{Three data types:\\ continuous, counts and coin flips}
\author{Cóilín Minto, Olga Lyashevska}
\date{July 15\textsuperscript{th} 2022}
\institute{Marine and Freshwater Research Centre\\ Atlantic Technological University \\ Galway, Ireland}
%% logo
\titlegraphic{
    \includegraphics[width=4cm]{figures/ATU-Logo-Full-RGB-Green-big.jpg}
}

\AtBeginSection[]{
\begin{frame}[noframenumbering, plain]
\frametitle{Outline}
\tableofcontents[currentsection]
\addtocounter{page}{-1}
\end{frame}
}

\begin{document}

\begin{frame}
 \maketitle
\end{frame}

\section{Data types}

%%------------------
%% CONTINUOUS DATA
%%------------------
\begin{frame}
 \frametitle{Describe some features of the response data $y$}
 \only<1>{
 \begin{center}
    \includegraphics[width=10cm]{figures/continuous_y_0.pdf}
 \end{center}
} 
 \only<2>{
 \textcolor{atugreen}{{\bf Continuous data}}\vspace{.5cm}
 \begin{itemize}
  \item Response $y$ is continuous, e.g., $y = 1.25$ possible
  \item Response can be positive or negative (on the real line)
  \item Apparent positive linear relationship with continuous variable $x$
  \item {\bf Example} $y$ could be a change in water height
 \end{itemize}
} 
\end{frame}

%% positive continuous
\begin{frame}
 \frametitle{Describe some features of the response data $y$}
 \only<1>{
 \begin{center}
    \includegraphics[width=10cm]{figures/continuous_y_1.pdf}
 \end{center}
} 
 \only<2>{
 \textcolor{Orange}{{\bf Positive continuous data}}\vspace{.5cm}  
 \begin{itemize}
  \item Response $y$ is also continuous, e.g., $y = 0.25$ possible
  \item Response can \underline{only} be positive (on the positive real line)
  \item Apparent positive non-linear relationship with continuous variable $x$
  \item {\bf Example} $y$ could be mass of individuals
   \begin{itemize}
    \item Discuss what values mass/weight of a fish could be
   \end{itemize}
 \end{itemize}
}
\end{frame}

%%------------------
%% COUNT DATA
%%------------------
\begin{frame}
 \frametitle{Describe some features of the response data $y$}
 \only<1>{
 \begin{center}
    \includegraphics[width=10cm]{figures/count_y.pdf}
 \end{center}
} 
 \only<2>{
  \textcolor{Salmon}{{\bf Count data}}\vspace{.5cm}
 \begin{itemize}
  \item Response $y$ is a count (discrete), e.g., $y = 1.25$ impossible
  \item Response can be zero or a positive integer
  \item Apparent positive non-linear relationship with continuous variable $x$
  \item {\bf Example} $y$ could be an organism count per unit area (abundance)
   \begin{itemize}
    \item Discuss what values of abundance are possible
   \end{itemize}  
 \end{itemize}
} 
\end{frame}

%%------------------
%% BINARY DATA
%%------------------
\begin{frame}
 \frametitle{Describe some features of the response data $y$}
 \only<1>{
 \begin{center}
    \includegraphics[width=10cm]{figures/binary_y.pdf}
 \end{center}
} 
 \only<2>{
  \textcolor{CornflowerBlue}{{\bf Binary data}}\vspace{.5cm}
 \begin{itemize}
  \item Response $y$ can be either a 1 or a 0 (or other binary categories, e.g., on/off) 
   \begin{itemize}
    \item Often it is a sum of positives out of a given number of trials, e.g., total number of heads in 10 coin flips
    \item Key thing is that for any one flip there can only be 2 outcomes   
   \end{itemize}  
  \item Apparent positive non-linear relationship with continuous variable $x$
  \item {\bf Example} $y$ could be maturity status (mature/immature) for an organism
   \begin{itemize}
    \item Discuss other binary data examples
   \end{itemize}  
 \end{itemize}
} 
\end{frame}



\section{Probability distributions}

\begin{frame}
 \frametitle{Probability distribution}
 A function that describes the probabilities associated with possible outcomes for an experiment (think of the response $y$)
\end{frame}

\begin{frame}
 \frametitle{Continuous probability distributions}
 \begin{center}
 \textcolor{atugreen}{Normal distribution}
 \end{center}
 \only<1>{
% \vspace{-1cm}
 \begin{center}
    \includegraphics[width=9cm]{figures/normal_0.pdf}
 \end{center}
}

 \begin{itemize}
  \item<only@2> Distribution is continuous, e.g., $y = 1.25$ possible
  \item Positive or negative values possible (on the real line)
  \item Governed by two parameters: mean $\mu$ and variance $\sigma^2$
  \item Write: $y \sim \textnormal{N}(\mu, \sigma^2)$
 \end{itemize}
\end{frame}

\begin{frame}
 \frametitle{Positive continuous probability distributions}
 \begin{center}
 \textcolor{Orange}{Lognormal distribution}
 \end{center}
 \only<1>{
% \vspace{-1cm}
 \begin{center}
    \includegraphics[width=9cm]{figures/lognormal_0.pdf}
 \end{center}
}

 \begin{itemize}
  \item<only@2> Distribution is continuous, e.g., $y = 1.25$ possible
  \item Only positive values possible (on the positive real line)
  \item Governed by two parameters: mean $\mu$ and standard deviation $\sigma$ (both on log scale)
  \item Write: $y \sim \textnormal{Lognormal}(\mu, \sigma)$
 \end{itemize}
\end{frame}

%% Poisson
\begin{frame}
 \frametitle{Count probability distributions}
 \begin{center}
 \textcolor{Salmon}{Poisson distribution}
 \end{center}
 \only<1>{
% \vspace{-1cm}
 \begin{center}
    \includegraphics[width=9cm]{figures/poisson_0.pdf}
 \end{center}
}

 \begin{itemize}
  \item<only@2> Distribution is discrete, e.g., $y = 1.25$ impossible
  \item Distribution is only positive at zero and positive integers
  \item Governed by one parameter: rate $\lambda$ (e.g., density)
  \begin{itemize}
  \item Discuss rates in relation to counts
  \end{itemize}
  \item Write: $y \sim \textnormal{Pois}(\lambda)$
 \end{itemize}
\end{frame}

%% Binary
\begin{frame}
 \frametitle{Binary probability distribution}
 \begin{center}
 \textcolor{CornflowerBlue}{Binary (Bernoulli) distribution}
 \end{center}
 \only<1>{
% \vspace{-1cm}
 \begin{center}
    \includegraphics[width=9cm]{figures/binary_0.pdf}
 \end{center}
}

\begin{itemize}
  \item<only@2> Distribution over 0 or 1 (or other binary categories) only
  \item Governed by parameter: probability of success $p$ (e.g., probability of being mature)
  \item Think: coin flip but coin not necessarily fair
  \item Write $y \sim \textnormal{Bernoulli}(p)$
 \end{itemize}
\end{frame}

%% Binomial
\begin{frame}
 \frametitle{Binomial probability distribution}
 \begin{center}
 \textcolor{CornflowerBlue}{Binomial distribution}
 \end{center}
 \only<1>{
% \vspace{-1cm}
 \begin{center}
    \includegraphics[width=9cm]{figures/binomial_0.pdf}
 \end{center}
}

\begin{itemize}
  \item<only@2> Distribution over $\{0, 1, \dots,n\}$ only
  \item Governed by 2 parameters: number of trials $n$ (think: number coin flips) and probability of success $p$ on any trial
  \item Write $y \sim \textnormal{Bin}(n,p)$
 \end{itemize}
 Note: Binomial is the sum of Bernoulli trials
\end{frame}

%%-----------------------
%% EXPLANATORY VARIABLES
%%-----------------------

\section{Explanatory variables}
\begin{frame}
 \frametitle{Explanatory variables\footnote{Also called ``predictors''}}
 Often a goal of an experiment or observational study is to relate observed response values to explanatory variables, e.g.,
 \begin{itemize}
  \item<2-> Does mass depend on length and is this the same across areas?
  \item<3-> Does abundance relate to environment, e.g., temperature?
  \item<4-> Does behavioural response depend on time-of-day?
  \item<5-> \dots
 \end{itemize}
\onslide<5->{We would like to explore/model the relationships between the response and explanatory variables\\}
\onslide<6>{Let's look at some \dots}
\end{frame}

\begin{frame}
 \frametitle{How well does the model describe the response data?}
 \begin{center}
    \includegraphics[width=10cm]{figures/continuous_xy_0.pdf}
 \end{center}
\end{frame}

\begin{frame}
 \frametitle{How well does the model describe the response data?}
 \begin{center}
    \includegraphics<1-1>[width=10cm]{figures/continuous_xy_0.pdf}
    \includegraphics<2-2>[width=10cm]{figures/continuous_xy_1.pdf}
    \includegraphics<3-3>[width=10cm]{figures/continuous_xy_2.pdf}
    \includegraphics<4-4>[width=10cm]{figures/count_xy_0.pdf}
    \includegraphics<5-5>[width=10cm]{figures/count_xy_1.pdf}
    \includegraphics<6-6>[width=10cm]{figures/binary_xy_0.pdf}
    \includegraphics<7-7>[width=10cm]{figures/binary_xy_1.pdf}
 \end{center}
\end{frame}

\begin{frame}
 \frametitle{Including explanatory variables}
 Need to explain variability in the response with explanatory variables, \underline{while respecting the distribution of the response}
\end{frame}

\section{Summary}
\begin{frame}
 \frametitle{Summary}
 \begin{itemize}
  \item<1-> Understanding the nature of the response data is key to success
  \item<2-> There are naturally suitable distributions for many data types
  \item<3-> Parameters of appropriate distributions can be related to explanatory variables
  \item<4-> Need a framework to address all of these requirements - it exists and is called GLM!
 \end{itemize}
\end{frame}

\begin{frame}
\begin{center}
Questions? 
\end{center}
\end{frame}
 



\end{document}

\begin{frame}
 \frametitle{}
\end{frame}
