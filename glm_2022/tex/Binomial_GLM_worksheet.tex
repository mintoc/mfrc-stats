\documentclass[a4paper,12pt]{article}
\usepackage[utf8x]{inputenc}
\usepackage[top=2cm, left=2cm, right=2cm, bottom=2cm]{geometry}
\usepackage{times}
\usepackage{enumitem}
\usepackage{hyperref}
\usepackage{color}
\usepackage{amsmath}
\usepackage{longtable}
\setitemize{itemsep=0pt, topsep=0pt}
\newcommand{\mydash}{\hspace{0.2cm}--\hspace{0.2cm}}
%opening
\title{Binomial GLM worksheet}
\author{C\'oil\'in Minto, Olga Lyashevska}
\date{}
\begin{document}

\maketitle

\noindent 
The length, sex and maturity status of 2,899 were measured during the Irish Groundfish Survey. The data are in the file \texttt{data/haddock\_maturity.csv} \href{https://github.com/mintoc/mfrc-stats}{here}.\\ 

Our main scientific question is whether the probability of being mature varies with length?

\vspace{1cm}
\begin{itemize}
 \item[{\bf 1.}] Read the data into R.
 \item[{\bf 2.}] Plot the data in an informative way - use colours!
 \item[{\bf 3.}] Use the \texttt{glm} function to fit a binomial GLM to with maturity as the response and fish length as the explanatory variable - remember to supply a matrix (one column for number of successes, another for the number of failures) for the binomial GLM in R.
 \item[{\bf 4.}] Plot the fitted values over your data plot.
 \item[{\bf 5.}] Bonus: Include sex in the model, refit and plot.
\end{itemize}

\end{document}
